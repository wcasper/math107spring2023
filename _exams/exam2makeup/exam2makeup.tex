% Exam Template for UMTYMP and Math Department courses
%
% Using Philip Hirschhorn's exam.cls: http://www-math.mit.edu/~psh/#ExamCls
%
% run pdflatex on a finished exam at least three times to do the grading table on front page.
%
%%%%%%%%%%%%%%%%%%%%%%%%%%%%%%%%%%%%%%%%%%%%%%%%%%%%%%%%%%%%%%%%%%%%%%%%%%%%%%%%%%%%%%%%%%%%%%

% These lines can probably stay unchanged, although you can remove the last
% two packages if you're not making pictures with tikz.
\documentclass[11pt]{exam}
\RequirePackage{amssymb, amsfonts, amsmath, latexsym, verbatim, xspace, setspace}
\RequirePackage{tikz, pgflibraryplotmarks}

% By default LaTeX uses large margins.  This doesn't work well on exams; problems
% end up in the "middle" of the page, reducing the amount of space for students
% to work on them.
\usepackage[margin=1in]{geometry}
\usepackage{enumerate}
\usepackage{amsthm}
\usepackage{listings}

\theoremstyle{definition}
\newtheorem{soln}{Solution}

\lstset{frame=tb,
  language=MATLAB,
  aboveskip=3mm,
  belowskip=3mm,
  showstringspaces=false,
  columns=flexible,
  basicstyle={\small\ttfamily},
  numbers=none,
  numberstyle=\tiny\color{gray},
  keywordstyle=\color{blue},
  commentstyle=\color{dkgreen},
  stringstyle=\color{mauve},
  breaklines=true,
  breakatwhitespace=true,
  tabsize=3
}

% Here's where you edit the Class, Exam, Date, etc.
\newcommand{\class}{Math 107 Section 2}
\newcommand{\term}{Spring 2022}
\newcommand{\examnum}{Exam II Makeup Written Portion}
\newcommand{\examdate}{March 2, 2022}
\newcommand{\timelimit}{50 Minutes}
\newcommand{\ol}[1]{\overline{#1}}

% For an exam, single spacing is most appropriate
\singlespacing
% \onehalfspacing
% \doublespacing

% For an exam, we generally want to turn off paragraph indentation
\parindent 0ex

\begin{document} 

% These commands set up the running header on the top of the exam pages
\pagestyle{head}
\firstpageheader{}{}{}
\runningheader{\class}{\examnum\ - Page \thepage\ of \numpages}{\examdate}
\runningheadrule

\begin{flushright}
\begin{tabular}{p{2.8in} r l}
\textbf{\class} & \textbf{Name (Print):} & \makebox[2in]{\hrulefill}\\
\textbf{\term} &&\\
\textbf{\examnum} & \textbf{Student ID:}&\makebox[2in]{\hrulefill}\\
\textbf{\examdate} &&\\
\textbf{Time Limit: \timelimit} % & Teaching Assistant & \makebox[2in]{\hrulefill}
\end{tabular}\\
\end{flushright}
\rule[1ex]{\textwidth}{.1pt}


This exam contains \numpages\ pages (including this cover page) and
\numquestions\ problems.  Check to see if any pages are missing.  Enter
all requested information on the top of this page, and put your initials
on the top of every page, in case the pages become separated.\\

You may \textit{not} use your books or notes on this exam.  However, you may use a \textit{basic} calculator.\\

You are required to show your work on each problem on this exam.  The following rules apply:\\

\begin{minipage}[t]{3.7in}
\vspace{0pt}
\begin{itemize}

%\item \textbf{If you use a ``fundamental theorem'' you must indicate this} and explain
%why the theorem may be applied.

\item \textbf{Organize your work}, in a reasonably neat and coherent way, in
the space provided. Work scattered all over the page without a clear ordering will 
receive very little credit.  

\item \textbf{Mysterious or unsupported answers will not receive full
credit}.  A correct answer, unsupported by calculations, explanation,
or algebraic work will receive no credit; an incorrect answer supported
by substantially correct calculations and explanations might still receive
partial credit.  This especially applies to limit calculations.

\item If you need more space, use the back of the pages; clearly indicate when you have done this.

\end{itemize}

Do not write in the table to the right.
\end{minipage}
\hfill
\begin{minipage}[t]{2.3in}
\vspace{0pt}
%\cellwidth{3em}
\gradetablestretch{2}
\vqword{Problem}
\addpoints % required here by exam.cls, even though questions haven't started yet.	
\gradetable[v]%[pages]  % Use [pages] to have grading table by page instead of question

\end{minipage}
\newpage % End of cover page

%%%%%%%%%%%%%%%%%%%%%%%%%%%%%%%%%%%%%%%%%%%%%%%%%%%%%%%%%%%%%%%%%%%%%%%%%%%%%%%%%%%%%
%
% See http://www-math.mit.edu/~psh/#ExamCls for full documentation, but the questions
% below give an idea of how to write questions [with parts] and have the points
% tracked automatically on the cover page.
%
%
%%%%%%%%%%%%%%%%%%%%%%%%%%%%%%%%%%%%%%%%%%%%%%%%%%%%%%%%%%%%%%%%%%%%%%%%%%%%%%%%%%%%%

\begin{questions}

\addpoints

\question[10]\mbox{}

Consider the following lines of MATLAB code.  Determine the final values of the variables $k$ and $x$.  Carefully show your work by filling in the missing values in the table below.  Note that not all rows will necessarily be used!

\vspace{0.5in}
\begin{lstlisting}
N = 3;
k = 1;
x = 0;

for k = -N:N
  if k > 1
    x = x*k;
  else
    x = x - k;
  end
end
\end{lstlisting}
\vspace{0.5in}

\begin{center}
\begin{tabular}{|c|c|c|}\hline
loop iteration & $k$ & $x$\\\hline
$1$ & $-3$ & $3$ \\\hline&&\\
$2$ &      &     \\\hline&&\\
$3$ & $-1$ &     \\\hline&&\\
$4$ &      & $6$ \\\hline&&\\
    &      & $5$ \\\hline&&\\
$6$ &      &     \\\hline&&\\
&&\\\hline&&\\
&&\\\hline&&\\
&&\\\hline&&\\
\end{tabular}
\end{center}

\vspace{0.3in}

\begin{itemize}
\item \textbf{Final $k$ value:}
\item \textbf{Final $x$ value:}
\end{itemize}

\newpage
\question[10]\mbox{}

Solve the following problems by hand and carefully show your work.

\begin{enumerate}[(a)]
\item Let $z=2-3i$ and $w=-5+2i$.  Rewrite the complex number $\frac{w}{z}$ in $a+ib$ form.
\vspace{2.7in}
\item Let $z = -2 + 2i$.  Rewrite $z$ in the form $z = re^{i\theta}$ for some real numbers $r>0$ and $0\leq \theta < 2\pi$.
\vspace{2.7in}
\item Calculate the value of $(-2+2i)^{20}$ in $a + ib$ form.
\end{enumerate}


\newpage
\question[10]\mbox{}

\begin{enumerate}[(a)]
\item Write down the value of a $2\times 2$ matrix $A$ corresponding to the transformation of the $x,y$ plane which reflects everything across the line $y=-x$.
\vspace{2in}
\item Describe the transformation which corresponds to the matrix

$$\left(\begin{array}{cc}
  0 &  1\\
 -1 &  0
\end{array}\right)$$
\vspace{3in}

\item Consider the matrices

$$A = \left[\begin{array}{ccc} -2 & -1 & 4\\ 1 & 0 & 1\end{array}\right]
\quad
B = \left[\begin{array}{ccc} 4 & 0 & 3\\ 1 & 0 & 2\\ 1 & 2 & -1\end{array}\right]$$

Compute $AB$ and $BA$.  If one of the products is undefined, explain why.
\end{enumerate}

\newpage
\question[10]\mbox{}

Consider the following situation

The cost of a ticket to the Aquarium is $7$ dollars for a child $12$ dollars for a senior and $20$ dollars for an adult.
On a particular day, the aquarium sells twice the number of children's tickets as the number of adult and senior tickets combined.
The total number of tickets sold that day is $300$ and the total revenue generated from the sale of tickets is $3200$ dollars.

\begin{enumerate}[(a)]
\item Set up a linear system of equations for the following story problem
\vspace{2in}
\item Write down an augmented matrix describing the linear system of equations in (a)
\vspace{1.5in}
\item Solve the linear system in (a) by row reducing the augmented matrix in (b)
\end{enumerate}

\newpage
\question[10]\mbox{}

For each of the following statements, write TRUE if the statement is always true or FALSE if the statement can be false.
There is no need to justify your answers.

\begin{enumerate}[(a)]
\item If a linear system of equations has more than one solution, then it must have infinitely many solutions.
\vspace{1.1in}
\item If $A$, $B$, and $C$ are nonzero $2\times 2$ matrices and $AB = AC$ then $B=C$.
\vspace{1.1in}
\item The linear system of equations corresponding to the following augmented matrix for three equations and three variables is inconsistent:

$$\left[\begin{array}{ccc|c}
2 & 3 & 7 & 1\\
4 & 2 & 3 & 9\\
0 & 0 & 0 & 2\\
\end{array}\right]$$
\vspace{1.1in}

\item If $A$ and $B$ are two $2\times 2$ matrices then $AB = BA$
\vspace{1.1in}

\item $3+2i$, $8i$ and $\sqrt{2}$ are all examples of complex numbers
\end{enumerate}


\newpage
\question[10]\mbox{}


Consider the following linear system of equations

$$\left\lbrace\begin{array}{cc}
x + 2y + 3z &= -3\\
2x + 4z     &=-6\\
4x +4y +  10z     &=-12
\end{array}\right.$$

\begin{enumerate}[(a)]
\item Rewrite this linear system as an augmented matrix
\vspace{1.5in}
\item Put the augmented matrix in row reduced echelon form
\vspace{4in}
\item Determine which variables are free and which variables are dependent.  How many solutions does this system have?
\end{enumerate}

\end{questions}

\end{document}


