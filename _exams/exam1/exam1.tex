% Exam Template for UMTYMP and Math Department courses
%
% Using Philip Hirschhorn's exam.cls: http://www-math.mit.edu/~psh/#ExamCls
%
% run pdflatex on a finished exam at least three times to do the grading table on front page.
%
%%%%%%%%%%%%%%%%%%%%%%%%%%%%%%%%%%%%%%%%%%%%%%%%%%%%%%%%%%%%%%%%%%%%%%%%%%%%%%%%%%%%%%%%%%%%%%

% These lines can probably stay unchanged, although you can remove the last
% two packages if you're not making pictures with tikz.
\documentclass[11pt]{exam}
\RequirePackage{amssymb, amsfonts, amsmath, latexsym, verbatim, xspace, setspace}
\RequirePackage{tikz, pgflibraryplotmarks}

% By default LaTeX uses large margins.  This doesn't work well on exams; problems
% end up in the "middle" of the page, reducing the amount of space for students
% to work on them.
\usepackage[margin=1in]{geometry}
\usepackage{enumerate}
\usepackage{amsthm}
\usepackage{listings}

\theoremstyle{definition}
\newtheorem{soln}{Solution}

\lstset{frame=tb,
  language=MATLAB,
  aboveskip=3mm,
  belowskip=3mm,
  showstringspaces=false,
  columns=flexible,
  basicstyle={\small\ttfamily},
  numbers=none,
  numberstyle=\tiny\color{gray},
  keywordstyle=\color{blue},
  commentstyle=\color{dkgreen},
  stringstyle=\color{mauve},
  breaklines=true,
  breakatwhitespace=true,
  tabsize=3
}

% Here's where you edit the Class, Exam, Date, etc.
\newcommand{\class}{Math 107 Section 1}
\newcommand{\term}{Spring 2022}
\newcommand{\examnum}{Exam I}
\newcommand{\examdate}{March 2, 2022}
\newcommand{\timelimit}{50 Minutes}
\newcommand{\ol}[1]{\overline{#1}}

% For an exam, single spacing is most appropriate
\singlespacing
% \onehalfspacing
% \doublespacing

% For an exam, we generally want to turn off paragraph indentation
\parindent 0ex

\begin{document} 

% These commands set up the running header on the top of the exam pages
\pagestyle{head}
\firstpageheader{}{}{}
\runningheader{\class}{\examnum\ - Page \thepage\ of \numpages}{\examdate}
\runningheadrule

\begin{flushright}
\begin{tabular}{p{2.8in} r l}
\textbf{\class} & \textbf{Name (Print):} & \makebox[2in]{\hrulefill}\\
\textbf{\term} &&\\
\textbf{\examnum} & \textbf{Student ID:}&\makebox[2in]{\hrulefill}\\
\textbf{\examdate} &&\\
\textbf{Time Limit: \timelimit} % & Teaching Assistant & \makebox[2in]{\hrulefill}
\end{tabular}\\
\end{flushright}
\rule[1ex]{\textwidth}{.1pt}


This exam contains \numpages\ pages (including this cover page) and
\numquestions\ problems.  Check to see if any pages are missing.  Enter
all requested information on the top of this page, and put your initials
on the top of every page, in case the pages become separated.\\

You may \textit{not} use your books or notes on this exam.  However, you may use a \textit{basic} calculator.\\

You are required to show your work on each problem on this exam.  The following rules apply:\\

\begin{minipage}[t]{3.7in}
\vspace{0pt}
\begin{itemize}

%\item \textbf{If you use a ``fundamental theorem'' you must indicate this} and explain
%why the theorem may be applied.

\item \textbf{Organize your work}, in a reasonably neat and coherent way, in
the space provided. Work scattered all over the page without a clear ordering will 
receive very little credit.  

\item \textbf{Mysterious or unsupported answers will not receive full
credit}.  A correct answer, unsupported by calculations, explanation,
or algebraic work will receive no credit; an incorrect answer supported
by substantially correct calculations and explanations might still receive
partial credit.  This especially applies to limit calculations.

\item If you need more space, use the back of the pages; clearly indicate when you have done this.

\end{itemize}

Do not write in the table to the right.
\end{minipage}
\hfill
\begin{minipage}[t]{2.3in}
\vspace{0pt}
%\cellwidth{3em}
\gradetablestretch{2}
\vqword{Problem}
\addpoints % required here by exam.cls, even though questions haven't started yet.	
\gradetable[v]%[pages]  % Use [pages] to have grading table by page instead of question

\end{minipage}
\newpage % End of cover page

%%%%%%%%%%%%%%%%%%%%%%%%%%%%%%%%%%%%%%%%%%%%%%%%%%%%%%%%%%%%%%%%%%%%%%%%%%%%%%%%%%%%%
%
% See http://www-math.mit.edu/~psh/#ExamCls for full documentation, but the questions
% below give an idea of how to write questions [with parts] and have the points
% tracked automatically on the cover page.
%
%
%%%%%%%%%%%%%%%%%%%%%%%%%%%%%%%%%%%%%%%%%%%%%%%%%%%%%%%%%%%%%%%%%%%%%%%%%%%%%%%%%%%%%

\begin{questions}

\addpoints

\question[10]\mbox{}

For this problem, consider the matrix

$$Q = \left(\begin{array}{ccc}
  8 &  6 & 7\\
  5 &  3 & 0\\
  9 &  x & \heartsuit\\
  f &  1 & 43\\
 \Delta & z &  \clubsuit
\end{array}\right)$$

a) Write down the values of $Q(5,1)$, $Q(3,3)$, and $Q(2,3)$.  Be sure to specify which is which.

b) Write down an MATLAB expression which will create the $4\times 2$ submatrix of $Q$ obtained by deleting the middle row and the middle column

c) Write down the value of the matrix $Q(1:2:\text{end},1:\text{end})$

\newpage
\question[10]\mbox{}

\begin{enumerate}[(a)]
\item Determine the final values of the variables $x$, $y$, and $z$ after the list of commands are executed in the command window.
You must show your work by hand.

\begin{lstlisting}
x = 3;
y = 1;
z = 9;
x = y-z;
z = x + 2*y;
y = 2*y-z + x;
\end{lstlisting}

\vspace{2in}
\item Consider the function defined below.

\begin{lstlisting}
function [a,b,c] = examFun(x,y)
a = 2*x - y
b = x^2-y
c = a + b
end
\end{lstlisting}

Determine the final values of the variables $x$, $y$, $z$, $a$, and $b$ after running the following lines in the command window.  You must show your work by hand.

\begin{lstlisting}
a = 1;
b = 4;
x = 2;
y = 3;
z = -1;
[z,y,b] = examFun(x,a)
\end{lstlisting}

\end{enumerate}

\newpage
\question[10]\mbox{}

\begin{enumerate}[(a)]
\item Find scalar values $a$ and $b$ satisfying

$$a\binom{2}{-3} + b\binom{1}{4} = \binom{8}{-23}.$$
\vspace{2in}

\item Find a vector $\vec v$ satisfying

$$3\binom{2}{5} -3\vec v = \binom{12}{18}$$
\vspace{2in}

\item Find a matrix $H$ such that

$$
2H
+
4\left(\begin{array}{cc}
1 & -2\\2 & 1
\end{array}\right)
=
2\left(\begin{array}{cc}
2 & 3\\-4 & 1
\end{array}\right)
$$

\end{enumerate}

\newpage
\question[10]\mbox{}
Consider the following code

\begin{lstlisting}
x = 1;
y = 1;
while(y < 30)
  z = x+y;
  x = y;
  y = z;
end
\end{lstlisting}

Determine the values of $x$ and $y$ at the end of the code block.

Be sure to carefully show your work!


\newpage
\question[10]\mbox{}
Without using any built-in MATLAB functions, write down a function called \textit{odd\_sum} which takes in a value $n$ and returns the sum of the odd integers between $1$ and $n$.
Make sure that your function is well-documented!


\end{questions}

\end{document}


