% Exam Template for UMTYMP and Math Department courses
%
% Using Philip Hirschhorn's exam.cls: http://www-math.mit.edu/~psh/#ExamCls
%
% run pdflatex on a finished exam at least three times to do the grading table on front page.
%
%%%%%%%%%%%%%%%%%%%%%%%%%%%%%%%%%%%%%%%%%%%%%%%%%%%%%%%%%%%%%%%%%%%%%%%%%%%%%%%%%%%%%%%%%%%%%%

% These lines can probably stay unchanged, although you can remove the last
% two packages if you're not making pictures with tikz.
\documentclass[11pt]{exam}
\RequirePackage{amssymb, amsfonts, amsmath, latexsym, verbatim, xspace, setspace}
\RequirePackage{tikz, pgflibraryplotmarks}

% By default LaTeX uses large margins.  This doesn't work well on exams; problems
% end up in the "middle" of the page, reducing the amount of space for students
% to work on them.
\usepackage[margin=1in]{geometry}
\usepackage{enumerate}
\usepackage{amsthm}
\usepackage{listings}

\theoremstyle{definition}
\newtheorem{soln}{Solution}

\lstset{frame=tb,
  language=MATLAB,
  aboveskip=3mm,
  belowskip=3mm,
  showstringspaces=false,
  columns=flexible,
  basicstyle={\small\ttfamily},
  numbers=none,
  numberstyle=\tiny\color{gray},
  keywordstyle=\color{blue},
  commentstyle=\color{dkgreen},
  stringstyle=\color{mauve},
  breaklines=true,
  breakatwhitespace=true,
  tabsize=3
}

% Here's where you edit the Class, Exam, Date, etc.
\newcommand{\class}{Math 107 Section 2}
\newcommand{\term}{Spring 2022}
\newcommand{\examnum}{Final Exam}
\newcommand{\examdate}{May 16, 2022}
\newcommand{\timelimit}{50 Minutes}
\newcommand{\ol}[1]{\overline{#1}}

% For an exam, single spacing is most appropriate
\singlespacing
% \onehalfspacing
% \doublespacing

% For an exam, we generally want to turn off paragraph indentation
\parindent 0ex

\begin{document} 

% These commands set up the running header on the top of the exam pages
\pagestyle{head}
\firstpageheader{}{}{}
\runningheader{\class}{\examnum\ - Page \thepage\ of \numpages}{\examdate}
\runningheadrule

\begin{flushright}
\begin{tabular}{p{2.8in} r l}
\textbf{\class} & \textbf{Name (Print):} & \makebox[2in]{\hrulefill}\\
\textbf{\term} &&\\
\textbf{\examnum} & \textbf{Student ID:}&\makebox[2in]{\hrulefill}\\
\textbf{\examdate} &&\\
\textbf{Time Limit: \timelimit} % & Teaching Assistant & \makebox[2in]{\hrulefill}
\end{tabular}\\
\end{flushright}
\rule[1ex]{\textwidth}{.1pt}


This exam contains \numpages\ pages (including this cover page) and
\numquestions\ problems.  Check to see if any pages are missing.  Enter
all requested information on the top of this page, and put your initials
on the top of every page, in case the pages become separated.\\

You may \textit{not} use your books or notes on this exam.  However, you may use a \textit{basic} calculator.\\

You are required to show your work on each problem on this exam.  The following rules apply:\\

\begin{minipage}[t]{3.7in}
\vspace{0pt}
\begin{itemize}

%\item \textbf{If you use a ``fundamental theorem'' you must indicate this} and explain
%why the theorem may be applied.

\item \textbf{Organize your work}, in a reasonably neat and coherent way, in
the space provided. Work scattered all over the page without a clear ordering will 
receive very little credit.  

\item \textbf{Mysterious or unsupported answers will not receive full
credit}.  A correct answer, unsupported by calculations, explanation,
or algebraic work will receive no credit; an incorrect answer supported
by substantially correct calculations and explanations might still receive
partial credit.  This especially applies to limit calculations.

\item If you need more space, use the back of the pages; clearly indicate when you have done this.

\end{itemize}

Do not write in the table to the right.
\end{minipage}
\hfill
\begin{minipage}[t]{2.3in}
\vspace{0pt}
%\cellwidth{3em}
\gradetablestretch{2}
\vqword{Problem}
\addpoints % required here by exam.cls, even though questions haven't started yet.	
\gradetable[v]%[pages]  % Use [pages] to have grading table by page instead of question

\end{minipage}
\newpage % End of cover page

%%%%%%%%%%%%%%%%%%%%%%%%%%%%%%%%%%%%%%%%%%%%%%%%%%%%%%%%%%%%%%%%%%%%%%%%%%%%%%%%%%%%%
%
% See http://www-math.mit.edu/~psh/#ExamCls for full documentation, but the questions
% below give an idea of how to write questions [with parts] and have the points
% tracked automatically on the cover page.
%
%
%%%%%%%%%%%%%%%%%%%%%%%%%%%%%%%%%%%%%%%%%%%%%%%%%%%%%%%%%%%%%%%%%%%%%%%%%%%%%%%%%%%%%

\begin{questions}

\addpoints

\question[10]\mbox{}

Consider the following lines of MATLAB code.  Determine the final values of the variables $k$ and $m$.  Carefully show your work by filling in the missing values in the table below.  Note that not all rows will necessarily be used!

\begin{lstlisting}
k = 13;
m = 0;

while k ~= 1
  if mod(k,2) == 1
    k = 3*k + 1;
  else
    k = k/2;
  end

  m = m + 1;
end
\end{lstlisting}

\vspace{0.5in}

\begin{center}
\begin{tabular}{|c|c|c|}\hline
loop iteration & $m$ & $k$\\\hline
 $1$ & $1$  & $40$ \\\hline&&\\
 $2$ & $2$  &      \\\hline&&\\
 $3$ &      &      \\\hline&&\\
 $4$ & $4$  &  $5$ \\\hline&&\\
 $5$ &      &      \\\hline&&\\
 $6$ &      &      \\\hline&&\\
 $7$ &&\\\hline&&\\
 $8$ &&\\\hline&&\\
 $9$ &&\\\hline&&\\
$10$ &&\\\hline&&\\
$11$ &&\\\hline&&\\
\end{tabular}
\end{center}

\vspace{0.3in}

\begin{itemize}
\item \textbf{Final $k$ value:}
\item \textbf{Final $x$ value:}
\end{itemize}


\newpage
\question[10]\mbox{}

For each of the following, write TRUE if the statement is true or FALSE if the statement is false.
No justification is required.

\begin{enumerate}[(a)]
\item  A homogeneous linear system of equations always has a solution.
\vspace{1.5in}
\item  If $x$ is a real number satisfying $e^{ix} = 0$, then $x= 0$.
\vspace{1.5in}
\item  If $A$ is any matrix and $A^T$ is its transpose, then the matrices $AA^T$ and $A^TA$ are the same.
\vspace{1.5in}
\item  The product of a complex number $z$ with its complex conjugate $\overline z$ is always real.
\vspace{1.5in}
\item  If the RGB values of a certain pixel are $(255,255,255)$ then that pixel is bright white.
\end{enumerate}

\newpage
\question[10]\mbox{}

\begin{enumerate}[(a)]
\item  Find the values of $a$ for which the following matrix is invertible

$$A = \left[\begin{array}{cc} 1 & 3\\a & 2\end{array}\right]$$
\vspace{1.5in}

\item  Give an example of two $2\times 2$ matrices $A$ and $B$ with $AB \neq BA$
\vspace{1in}

\item  Calculate explicitly the value of the inverse of

$$A = \left(\begin{array}{ccc}
7 & 2 & 1\\ 0 & 3 & -1\\ -3 & 4 & -2\
\end{array}\right).$$

Then use the value of the inverse to find a solution of the linear system of equations

$$\left\lbrace\begin{array}{cc}
7x + 2y + z &= 21\\
3y-z &= 5\\
-3x + 4y-2z &= -1
\end{array}\right.$$
\end{enumerate}

\newpage
\question[10]\mbox{}
Consider the matrix

$$A = \left[\begin{array}{cc}5 & 2\\ -3 & 0\end{array}\right]$$

\begin{enumerate}[(a)]
\item Calculate the eigenvalues of $A$ by hand.  Carefully show your work.
\vspace{2.5in}
\item For each eigenvalue of $A$, find all eigenvectors with that eigenvalue by hand.
\end{enumerate}

\newpage
\question[10]\mbox{}
\begin{enumerate}[(a)]
\item Find a complex number $z$ different from $1$ satisfying $z^7=1$.  [Hint: think about Euler's formula]
\vspace{2.0in}
\item On a particularly busy day a catering business sold $85$ sandwiches, $65$ bags of chips, and $210$ cookies for a lunch event.  They observed that men each ate $2$ sandwiches, $1$ bag of chips, and $4$ cookies; women ate $1$ sandwich, $1$ bag of chips, and $2$ cookies; kids ate half a sandwich, a bag of chips, and $3$ cookies.  Set up, but do not solve, the linear system of equations described by this story problem.
\vspace{2.0in}
\item Give an example of a linear system of equations with no solutions
\vspace{1.0in}
\item Give an example of a linear system of equations with infinitely many solutions
\vspace{1.0in}
\item Describe in your own words one application of eigenvectors and eigenvalues 
\end{enumerate}
\end{questions}

\end{document}


